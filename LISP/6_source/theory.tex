\section{Теоритическая часть.}

\newline
\underline{Способы определения функции в LISP.}
\begin{enumerate}
\item Макроопределение DEFUN - форма, которая особым образом обрабатывает некоторые из своих аргументов (DEFUN имя (список аргументов) (тело));

\item $\lambda$ выражение - способ записи выражения, когда вначале записывается формула, которую надо вычислить, а затем значения, над которыми её вычислить. 
(LAMBDA (аргументы) (тело))
\begin{enumerate}
\item "тяжеловесный" способ использования $\lambda$ - выражений, возник из лямбда-исчисления Чёрча из математики. ((LAMBDA (x,y) (+ x y)) 2 5) - дает 7;
\item использование APPLY особой функции называемой функционалом. Требуется #. (APPLY #'name (параметры)) (APPLY #'(LAMBDA (x y) (+ x y)) (2 5)). $\lambda$-описание часто используется при работе со структурированными списками, обходить которые нужно рекурсивно.
\end{enumerate}
\end{enumerate}

\underline{Вызов функции и блокировка.}
\newline
Вызов функции производится с помощью функционала APPLY - функционал, который применяет аргумент к остальным аргументам - приминяющий функционал. Принимает ровно 2 аргумента - функциональный и список произвольной длины. Применяет функциональный элемент к списку, что приводит к вычислению функции, заданной первым аргументом, со списком параметров, заданным вторым аргументом. Происходит  «вызов функции».
\newline
$#'$ - «function name» - reader macros - результатом вычисления является функция-объект, к которой можно применить APPLY или FUNCALL.
\newline
Функционалы - особые функции, формы, которые в качестве аргумента принимают другие функции (применяющие и отображающие).
\newline
# - функциональная блокировка
\newline
' - блокировка вычислений
\newline
Функциональная блокировка ~ FUNCTION. С её помощью можно зафиксировать контекст определения функции. В случае если функция не имеет свободных переменных, эти два варианта блокирования не отличаются в процессе обработки. (FUNCALL #'FUN ARG1 ARG2 ...) - FUN можно вычислять. Поскольку вычислениям подвергаются символьные выражения, этот функционал позволяет динамично строить выражения функции, применяя её к указанным аргументам. Единственное требование - после вычисления первого аргумента выражение должно представлять собой функции. Результат обработки не может быть макросом. FUNCALL является противоположностью FUNCTION.

\newline
\underline{Глобальные и локальные символьные атомы.}


Lisp - интерпритатор в ходе своей работы поддерживаеь специальную таблицу символьных атомов - таблицу символов, в которой хранится информация обо всех атомах, встретившихся в тексте интерпритируемой программы или используемых в качестве обрабатываемых данных. Для присваивания рабочей переменной значения применяется форма SET. Чтобы присвоить переменной pi значение 22/7: 
\begin{lstlisting} (SET (QUOTE PI) 22/7) \end{lstlisting}
или
\begin{lstlisting} (SETQ PI 22/7) \end{lstlisting}



В общем случае атом имеет несколько разных значений. Они независимы друг от
друга, один и тот же символьный атом может быть использован как имя функции и как имя формального параметра. Нужная его интерпретация обеспечивается, исходя из контекста его применения. В таблице символьных атомов для каждого атома хранятся не сами значения, а указатели на внутренние представления 
этих значений. Внутренним представлением символьного атома является указатель на соответсвующий элемент таблицы атомов. 


Глобальные атомы определяются в глобальной области видимости. 
Локальные атомы определяются в области видимости функции. 
\newline