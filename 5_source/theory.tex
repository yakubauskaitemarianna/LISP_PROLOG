\section{Теоритическая часть}
\newline
\underline{Атом. Представление в памяти}


Атом - символ, идентификатор - синтаксически - набор литер (букв латинского алфавита, цифр),
начинающихся с букв; специальные символы {T, Nil} - используются для обозначения логических констант;
самоопределяемые атомы - натуральные, дробные числа, вещественные числа, строки - последовательность
символов, заключенная в двойные апострофы. У каждого символьного атома есть окружение и время жизни, 
символьные атомы не дублируются и в памяти представлены в виде 5 указателей:
\begin{enumerate}
	\item имя - name. В таблице символьных атомов хранится его внешнее имя, строка, представляющая его при вводе, выводе;
	\item значение - value. В таблице символьных атомов хранится его значение, как аргумента функции, фактическое значение, которое с ним связано, при его использовании в роли формального параметра некоторой функции;
	\item функция - function. В таблице символьных атомов хранится его функциональное значение, определяющее выражение функции при его использовании в качетсве имени этой функции;
	\item свойство - properties. В таблице символьных атомов хранится список свойств атома за исключением внешнего имени, остальные значения могут отсутсвовать;
	\item пакет - package - логические пакеты - таблицы, отображающие имена в символы.
\end{enumerate}

\newline
\underline{Локальное и глобальное определение значения атома.}


Lisp - интерпритатор в ходе своей работы поддерживаеь специальную таблицу символьных атомов - таблицу символов, в которой хранится информация обо всех атомах, встретившихся в тексте интерпритируемой программы или используемых в качестве обрабатываемых данных. Для присваивания рабочей переменной значения применяется форма SET. Чтобы присвоить переменной pi значение 22/7: 
\begin{lstlisting} (SET (QUOTE PI) 22/7) \end{lstlisting}
или
\begin{lstlisting} (SETQ PI 22/7) \end{lstlisting}



В общем случае атом имеет несколько разных значений. Они независимы друг от
друга, один и тот же символьный атом может быть использован как имя функции и как имя формального параметра. Нужная его интерпретация обеспечивается, исходя из контекста его применения. В таблице символьных атомов для каждого атома хранятся не сами значения, а указатели на внутренние представления 
этих значений. Внутренним представлением символьного атома является указатель на соответсвующий элемент таблицы атомов. 


Глобальные атомы определяются в глобальной области видимости. 
Локальные атомы определяются в области видимости функции. 

\newline
\underline{EVAL} - функция, которая принудительно вычисляет значения, обеспечивает дополнительный вызов интерпретатора LISP, вызов может производится внутри вычисляемого S-выражения, позволяет снять блокировку QUOTE.

\newline
\underline{QUOTE} - специальная одноаргументая функция, которая блокирует вычисления и возвращает в качестве значения этот аргумент.
